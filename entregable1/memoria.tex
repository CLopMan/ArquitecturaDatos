\documentclass[]{article}
\usepackage{graphicx}
\usepackage[spanish]{babel}
\usepackage[a4paper, top=2.5cm, bottom=2.5cm, left=3cm, right=3cm]{geometry}
\usepackage[hidelinks]{hyperref}
\usepackage{listings}
\usepackage{xcolor}

% style for listings (código)
\lstdefinestyle{mystyle} {
    commentstyle=\color{gray},
    keywordstyle=\color{blue},
    numberstyle=\color{yellow},
    stringstyle=\color{brown},
    breaklines=true,
    breakatwhitespace=false,
    captionpos=b,
    keepspaces=true,
    numbers=none,
    numbersep=5pt,
    showspaces=false,
    showtabs=false,
    tabsize=2
}
\lstset{style=mystyle}

%title
\title{Práctica 1} 

\author{Adrián Ferández Galán, César López Mantecón y Manuel Gómez-Plana Rodríguez}

\begin{document}

\begin{titlepage}
    \centering
   \includegraphics[width=0.9\textwidth]{uc3m.jpg} 
    {\Huge Universidad Carlos III\\
    
     \Large Arquitectura de Datos\\
     \vspace{0.5cm}
     Curso 2024-25}
    \vspace{2cm}

    {\Huge \textbf{Práctica 1} \par}
    \vspace{0.5cm}
    {\Large Diseño de una Base de Datos no relacional \par}
    \vspace{8cm}

   \textbf{Ingeniería Informática, Cuarto curso}\\
    \vspace{0.2cm} 
    Adrián Fernández Galán       (NIA: 100472182, e-mail: 100472182@alumnos.uc3m.es) \\
    César López Mantecón         (NIA: 100472092, e-mail: 100472092@alumnos.uc3m.es)\\
    Manuel Gómez-Plana Rodríguez (NIA: 100472092, e-mail: 100472092@alumnos.uc3m.es)
    \vspace{0.5cm}

   
    \textbf{Prof .} Lourdes Moreno López\\
    
    \textbf{Grupo: } 81   
    
\end{titlepage}
\newpage

\renewcommand{\contentsname}{\centering Índice}
\tableofcontents

\newpage

\section{Introducción}
\label{sec:introduccion}


\section{Diseño conceptual: diagrama de clases UML}
\label{sec:disenno}
[Imagen del Modelo sin Agregados]

\begin{itemize}
    \item Explicación del modelo básico sin los casos de uso y sin destacar las relaciones entre clases
    \begin{itemize}
        \item Área (Centrándose en Tipo, lista de juegos, estado, N-juegos, clima)
        \item Juego (Centrándose en Estado, Modelo, Historial de intervenciones, lista de incidencias)
    \end{itemize}
    \item Análisis de los casos de uso y destacar decisiones en el diseño según estos casos
    \begin{itemize}
        \item Caso de uso A
        \begin{itemize}
            \item Este caso de uso está centrado en proporcionar un listado completo de los juegos instalados en las diferentes áreas
            \item Para poder listar las áreas dado un distrito según el Req 1 se ha creado una entidad Distrito que facilite la búsqueda
            \item Para satisfacer el Req 4 se ha añadido a la entidad Juego el atributo "patrón de desgaste" 
        \end{itemize}
        \item Caso de uso B
        \begin{itemize}
            \item Este Caso de uso está centrado en la capacidad del sistema de reportar incidencias
            \item Se ha incorporado una entidad Incidencia con las características asociadas a los requisitos y otra entidad Usuarios
            \item Para Req 2 se ha integrado en la entidad Incidencia el atributo lista de destinatarios, para conocer los usuarios que han realizado los reportes
            \item Habría que explicar el por qué de todos los atributos de la entidad Incidencia, ahora mismo no caigo 
        \end{itemize}
        \item Caso de Uso C
        \begin{itemize}
            \item Este caso de uso está centrado en analizar las condiciones metereológicas para realizar una planificación eficiente del mantenimiento
            \item Para poder cubrir Req 1 se ha creado una entidad clima con los atributos "Temperatura", "Precipitación" y "Fecha"
            \item Para Req 2 y Req 3 no es necesario realizar ningún cambio dado que se conocen los aspectos meteorológicos de las áreas y por lo tanto de sus juegos
            \item Req 4 queda fuera de nuestro alcance como diseñadores de bases de datos
        \end{itemize}
        \item Caso de Uso D
        \begin{itemize}
            \item Este caso de uso busca mejorar la toma de decisiones sobre las áreas recreativas a través de la capacidad de generar informes y realizar encuestas de satisfacción por parte de los usuarios
            \item Para cubrir Req 1 no es necesario introducir nuevas características ya que se contempló con anterioridad el uso incidencias
            \item Se ha incorporado una entidad Encuesta para satisfacer Req 2, esta entidad tiene el usuario que lo ha realizado, la satisfación que tiene con el juego y un comentario. 
            \item Para Req 3 proporcionamos la información de las áreas junto a los diferentes reportes
        \end{itemize}
        \item Caso de Uso E
        \begin{itemize}
            \item Para Req 1 no es necesario incluir ninguna característica, ya que los cambios realizados para el caso de uso A ya contemplan este requisito
            \item Para Req 2 se ha añadido el atributo "N-juegos" (esto no se podría calcular sin necesidad de tenerlo estático??)
            \item Req 3 es viable hacerlo??
            \item Req 4 queda fuera del nuestro alcance
        \end{itemize}
    \end{itemize}
\end{itemize}

\section{Diseño de agregados}
\label{sec:agregados}
\begin{itemize}
    \item Analizar las acciones (lectura, escritura y actualizaciones) de los casos de uso
    \item El caso de Uso A es completamente de lectura, ya que solo se quiere obtener información de las áreas y juegos
\end{itemize}

\section{Validación del esquema}
\label{sec:esquema}
Para mantener la integridad de los datos, MongoDB ofrece el uso de estrategias como el uso de referencias en vez de embeber los datos. Sin embargo, para aquellos datos que hemos decidido embeber, debemos redactar una serie de reglas para mantener la consistencia de nuestro esquema. Estas reglas son:
\begin{itemize}
    \item Nombres de \textit{Distrito}: Cada distrito debe tener un nombre único para así evitar duplicados.
    \item Coordenadas de \textit{Área}: Las coordenadas de las áreas serán una tupla que contenga en grados, minutos y segundos la latitud y longitud del área.
    \item Estado Operativo \textit{Área}: El estado operativo de las áreas sólo podrán guardar los valores "Operativas", "En Mantenimiento" y "Fuera de servicio".
    \item Número de Juegos \textit{Área}: El número de juegos instalados en un área debe ser un entero no negativo y siempre menor o igual al número máximo de juegos de un área.
    \item Número Máximo de Juegos \textit{Área}: El número máximo de juegos de un área debe ser un enterno no negativo.
    \item Temperatura \textit{Clima}: La temperatura de \textit{Clima} es un float y se medirá en grados Celsius.
    \item Precipitación \textit{Clima}: La precipitación de \textit{Clima} es un float y se medirá en litros por metro cuadrado.
    \item Estado Operativo \textit{Juego}: El estado operativo de los juegos sólo podrá guardar dos valores "Disponible" o "No Disponible".
    \item QR \textit{Juego}: El QR de los juegos debe de ser una imagen.
    \item Accesibilidad \textit{Juego}: La accesibilidad de un juego debe ser un texto que indique las personas con alguna deficiencia física, mental o sensorial que puedan acceder.
    \item Fechas \textit{Incidencia}: La fecha de apertura debe ser anterior a la fecha de cierre.
    \item Fechas: Las fechas de todas las entidades deben seguir el formato: "yyyy/mm/dd hh:MM:ss".
\end{itemize}



\end{document}
