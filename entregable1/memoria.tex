\documentclass[]{article}
\usepackage{graphicx}
\usepackage[spanish]{babel}
\usepackage[a4paper, top=2.5cm, bottom=2.5cm, left=3cm, right=3cm]{geometry}
\usepackage[hidelinks]{hyperref}
\usepackage{listings}
\usepackage{xcolor}

% style for listings (código)
\lstdefinestyle{mystyle} {
    commentstyle=\color{gray},
    keywordstyle=\color{blue},
    numberstyle=\color{yellow},
    stringstyle=\color{brown},
    breaklines=true,
    breakatwhitespace=false,
    captionpos=b,
    keepspaces=true,
    numbers=none,
    numbersep=5pt,
    showspaces=false,
    showtabs=false,
    tabsize=2
}
\lstset{style=mystyle}

%title
\title{Práctica 1} 

\author{Adrián Ferández Galán, César López Mantecón y Manuel Gómez-Plana Rodríguez}

\begin{document}

\begin{titlepage}
    \centering
   \includegraphics[width=0.9\textwidth]{uc3m.jpg} 
    {\Huge Universidad Carlos III\\
    
     \Large Arquitectura de Datos\\
     \vspace{0.5cm}
     Curso 2024-25}
    \vspace{2cm}

    {\Huge \textbf{Práctica 1} \par}
    \vspace{0.5cm}
    {\Large Diseño de una Base de Datos no relacional \par}
    \vspace{8cm}

   \textbf{Ingeniería Informática, Cuarto curso}\\
    \vspace{0.2cm} 
    Adrián Fernández Galán       (NIA: 100472182, e-mail: 100472182@alumnos.uc3m.es) \\
    César López Mantecón         (NIA: 100472092, e-mail: 100472092@alumnos.uc3m.es)\\
    Manuel Gómez-Plana Rodríguez (NIA: 100472092, e-mail: 100472092@alumnos.uc3m.es)
    \vspace{0.5cm}

   
    \textbf{Prof .} Lourdes Moreno López\\
    
    \textbf{Grupo: } 81   
    
\end{titlepage}
\newpage

\renewcommand{\contentsname}{\centering Índice}
\tableofcontents

\newpage

\section{Introducción}
\label{sec:introduccion}

\section{Diseño conceptual: diagrama de clases UML}
\label{sec:disenno}
En esta sección se describe el primer modelo conceptual teniendo en cuenta tanto la semántica extraída de la descripción de la práctica como de los casos de uso.

Podemos extraer la existencia de 8 entidades importantes en nuestro modelo:
\begin{enumerate}
    \item \textbf{Distrito:} Aglomeración de \textit{áreas recreativas}.
    \item \textbf{Área:} Representación de cada una de las áreas recreativas. Deben contener información sobre el clima, los \textit{juegos} que contiene, su estdo y su accesibilidad; entre otros.
    \item \textbf{Juego:} Representación de cada uno de los instrumentos en un \textit{área recreativa}. Contiene información sobre el modelo, estado, patrones de desgaste, etc.
    \item \textbf{Clima:} Contiene información meteorológica para una fecha.
    \item \textbf{Historial de intervenciones:} Matiene registro de las \textit{intervenciones} realizadas en un juego.
    \item \textbf{Intervención:} Representa una revisión o intervención sobre un \textit{juego}. Debe contener información sobre la fecha y observaciones realizadas sobre el \textit{juego}.
    \item \textbf{Usuario:} Contiene información sobre un usuario como la información de contacto.
    \item \textbf{Incidencia:} Contiene la información relativa a una incidencia sobre un \textit{juego}. Esto es, naturaleza de la incidencia, lista de usuarios que han reportado la incidencia, estado de la incidencia, empresa encargada de solucionarla, fecha de apertura, fecha de cierre y creador. 
    \item \textbf{Encuesta:} Registra información sobre la satisfacción de los \textit{usuarios} para un \textit{juego}.
\end{enumerate}
Con todo lo anterior, hemos realizado el siguiente diseño del sistema.

[Imagen del Modelo sin Agregados]

\subsection{Caso de uso A}
\label{subsec:casoA}
Este caso se centra en proporcionar un listado completo sobre los juegos instalados en diferentes áreas permitiendo una búsqueda por barrio, distrito o área. Por esto se incluye la entidad \textit{distito} y el campo \textit{barrio} en \textit{área}.

\begin{itemize}
    \item La base de datos mantendrá registros sobre todas las incidencias. No obstante la generación de informes es algo que queda fuera del diseño y se contempla como funcionalidad de la aplicación..
    \item 
\end{itemize}

\begin{itemize}
    \item Explicación del modelo básico sin los casos de uso y sin destacar las relaciones entre clases
    \begin{itemize}
        \item Área (Centrándose en Tipo, lista de juegos, estado, N-juegos, clima)
        \item Juego (Centrándose en Estado, Modelo, Historial de intervenciones, lista de incidencias)
    \end{itemize}
    \item Análisis de los casos de uso y destacar decisiones en el diseño según estos casos
    \begin{itemize}
        \item Caso de uso A
        \begin{itemize}
            \item Este caso de uso está centrado en proporcionar un listado completo de los juegos instalados en las diferentes áreas
            \item Para poder listar las áreas dado un distrito según el Req 1 se ha creado una entidad Distrito que facilite la búsqueda
            \item Para satisfacer el Req 4 se ha añadido a la entidad Juego el atributo "patrón de desgaste" 
        \end{itemize}
        \item Caso de uso B
        \begin{itemize}
            \item Este Caso de uso está centrado en la capacidad del sistema de reportar incidencias
            \item Se ha incorporado una entidad Incidencia con las características asociadas a los requisitos y otra entidad Usuarios
            \item Para Req 2 se ha integrado en la entidad Incidencia el atributo lista de destinatarios, para conocer los usuarios que han realizado los reportes
            \item Habría que explicar el por qué de todos los atributos de la entidad Incidencia, ahora mismo no caigo 
        \end{itemize}
        \item Caso de Uso C
        \begin{itemize}
            \item Este caso de uso está centrado en analizar las condiciones metereológicas para realizar una planificación eficiente del mantenimiento
            \item Para poder cubrir Req 1 se ha creado una entidad clima con los atributos "Temperatura", "Precipitación" y "Fecha"
            \item Para Req 2 y Req 3 no es necesario realizar ningún cambio dado que se conocen los aspectos meteorológicos de las áreas y por lo tanto de sus juegos
            \item Req 4 queda fuera de nuestro alcance como diseñadores de bases de datos
        \end{itemize}
        \item Caso de Uso D
        \begin{itemize}
            \item Este caso de uso busca mejorar la toma de decisiones sobre las áreas recreativas a través de la capacidad de generar informes y realizar encuestas de satisfacción por parte de los usuarios
            \item Para cubrir Req 1 no es necesario introducir nuevas características ya que se contempló con anterioridad el uso incidencias
            \item Se ha incorporado una entidad Encuesta para satisfacer Req 2, esta entidad tiene el usuario que lo ha realizado, la satisfación que tiene con el juego y un comentario. 
            \item Para Req 3 proporcionamos la información de las áreas junto a los diferentes reportes
        \end{itemize}
        \item Caso de Uso E
        \begin{itemize}
            \item Para Req 1 no es necesario incluir ninguna característica, ya que los cambios realizados para el caso de uso A ya contemplan este requisito
            \item Para Req 2 se ha añadido el atributo "N-juegos" (esto no se podría calcular sin necesidad de tenerlo estático??)
            \item Req 3 es viable hacerlo??
            \item Req 4 queda fuera del nuestro alcance
        \end{itemize}
    \end{itemize}
\end{itemize}

\section{Diseño de agregados}
\label{sec:agregados}
\begin{itemize}
    \item Analizar las acciones (lectura, escritura y actualizaciones) de los casos de uso
    \item El caso de Uso A es completamente de lectura, ya que solo se quiere obtener información de las áreas y juegos
\end{itemize}

\section{Validación del esquema}
\label{sec:esquema}
[escribe aquí]

\end{document}
