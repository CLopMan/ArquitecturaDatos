\documentclass[]{article}
\usepackage{graphicx}
\usepackage[spanish]{babel}
\usepackage[a4paper, top=2.5cm, bottom=2.5cm, left=3cm, right=3cm]{geometry}
\usepackage[hidelinks]{hyperref}
\usepackage{listings}
\usepackage{xcolor}

% style for listings (código)
\lstdefinestyle{mystyle} {
    commentstyle=\color{gray},
    keywordstyle=\color{blue},
    numberstyle=\color{yellow},
    stringstyle=\color{brown},
    breaklines=true,
    breakatwhitespace=false,
    captionpos=b,
    keepspaces=true,
    numbers=none,
    numbersep=5pt,
    showspaces=false,
    showtabs=false,
    tabsize=2
}
\lstset{style=mystyle}

%title
\title{Práctica 1} 

\author{Adrián Ferández Galán, César López Mantecón y Manuel Gómez-Plana Rodríguez}

\begin{document}

\begin{titlepage}
    \centering
   \includegraphics[width=0.9\textwidth]{uc3m.jpg} 
    {\Huge Universidad Carlos III\\
    
     \Large Arquitectura de Datos\\
     \vspace{0.5cm}
     Curso 2024-25}
    \vspace{2cm}

    {\Huge \textbf{Práctica 1} \par}
    \vspace{0.5cm}
    {\Large Diseño de una Base de Datos no relacional \par}
    \vspace{8cm}

   \textbf{Ingeniería Informática, Cuarto curso}\\
    \vspace{0.2cm} 
    Adrián Fernández Galán       (NIA: 100472182, e-mail: 100472182@alumnos.uc3m.es) \\
    César López Mantecón         (NIA: 100472092, e-mail: 100472092@alumnos.uc3m.es)\\
    Manuel Gómez-Plana Rodríguez (NIA: 100472092, e-mail: 100472092@alumnos.uc3m.es)
    \vspace{0.5cm}

   
    \textbf{Prof .} Lourdes Moreno López\\
    
    \textbf{Grupo: } 81   
    
\end{titlepage}
\newpage

\renewcommand{\contentsname}{\centering Índice}
\tableofcontents

\newpage

\section{Introducción}
\label{sec:introduccion}

\section{Diseño conceptual: diagrama de clases UML}
\label{sec:disenno}
[Imagen del Modelo sin Agregados]

\begin{itemize}
    \item Explicación del modelo básico sin los casos de uso y sin destacar las relaciones entre clases
    \begin{itemize}
        \item Área (Centrándose en Tipo, lista de juegos, estado, N-juegos, clima)
        \item Juego (Centrándose en Estado, Modelo, Historial de intervenciones, lista de incidencias)
    \end{itemize}
    \item Análisis de los casos de uso y destacar decisiones en el diseño según estos casos
    \begin{itemize}
        \item Caso de uso A
        \begin{itemize}
            \item Este caso de uso está centrado en proporcionar un listado completo de los juegos instalados en las diferentes áreas
            \item Para poder listar las áreas dado un distrito según el Req 1 se ha creado una entidad Distrito que facilite la búsqueda
            \item Para satisfacer el Req 4 se ha añadido a la entidad Juego el atributo "patrón de desgaste" 
        \end{itemize}
        \item Caso de uso B
        \begin{itemize}
            \item Este Caso de uso está centrado en la capacidad del sistema de reportar incidencias
            \item Se ha incorporado una entidad Incidencia con las características asociadas a los requisitos y otra entidad Usuarios
            \item Para Req 2 se ha integrado en la entidad Incidencia el atributo lista de destinatarios, para conocer los usuarios que han realizado los reportes
            \item Habría que explicar el por qué de todos los atributos de la entidad Incidencia, ahora mismo no caigo 
        \end{itemize}
        \item Caso de Uso C
        \begin{itemize}
            \item Este caso de uso está centrado en analizar las condiciones metereológicas para realizar una planificación eficiente del mantenimiento
            \item Para poder cubrir Req 1 se ha creado una entidad clima con los atributos "Temperatura", "Precipitación" y "Fecha"
            \item Para Req 2 y Req 3 no es necesario realizar ningún cambio dado que se conocen los aspectos meteorológicos de las áreas y por lo tanto de sus juegos
            \item Req 4 queda fuera de nuestro alcance como diseñadores de bases de datos
        \end{itemize}
        \item Caso de Uso D
        \begin{itemize}
            \item Este caso de uso busca mejorar la toma de decisiones sobre las áreas recreativas a través de la capacidad de generar informes y realizar encuestas de satisfacción por parte de los usuarios
            \item Para cubrir Req 1 no es necesario introducir nuevas características ya que se contempló con anterioridad el uso incidencias
            \item Se ha incorporado una entidad Encuesta para satisfacer Req 2, esta entidad tiene el usuario que lo ha realizado, la satisfación que tiene con el juego y un comentario. 
            \item Para Req 3 proporcionamos la información de las áreas junto a los diferentes reportes
        \end{itemize}
        \item Caso de Uso E
        \begin{itemize}
            \item Para Req 1 no es necesario incluir ninguna característica, ya que los cambios realizados para el caso de uso A ya contemplan este requisito
            \item Para Req 2 se ha añadido el atributo "N-juegos" (esto no se podría calcular sin necesidad de tenerlo estático??)
            \item Req 3 es viable hacerlo??
            \item Req 4 queda fuera del nuestro alcance
        \end{itemize}
    \end{itemize}
\end{itemize}

\section{Diseño de agregados}
\label{sec:agregados}
Con el objetivo de poder analizar las capacidades que tendrán los agregados es necesario conoecer las acciones que conllevan los distintos casos de uso.
\begin{itemize}
    \item \textbf{Lectura}: Casos de Uso A y E
    \item \textbf{Escritura}: Casos de Uso B, C y D
\end{itemize}

En el diseño de agregados para el sistema de áreas y sus juegos sea han creado 4 agregados, cada uno optimizado para casos de uso específicos. En estos agregados se ha buscado un equilibrio entre lecturas y escrituras para aquellos agregados enfocados a añadir elementos.

Además hemos determinado si las relaciones entre entidades deben de ser embebidas, referencias o tablas resumen. Estas decisiones se fundamentan en la frecuencia de lectura, modificación y crecimiento de los datos.

A continuación se describirán los agregados y las entidates que lo conforman, además de abordar las características de los agregados, como son la raíz y el perímetro.

\subsection{Agregado_Rojo}
\label{sub_sec:agregado_rojo}
\begin{itemize}
    \item \textbf{Entidades}: Distrito, Área y Juego
    \item \textbf{Raíz}: Distrito
    \item \textbf{Perímetro}: Este agregado está enfocado en poder consultar la información relevante sobre las áreas y los juegos que lo comprenden. Cada juego está asociado al área al que pertenece.
    \item \textbf{Casos de Uso optimizados}:
    \begin{itemize}
        \item CU\_A : (Listado detallado de juegos y su estado): Este agregado proporciona consultas rápidas para obtener información sobre los juegos y su estado actual.
        \item CU\_E : (Informe agrupado por distritos): También es capaz de generar un informe agrupado por distritos que muestre información sobre sus juegos. 
    \end{itemize}
\end{itemize}

[Imagen con agregados]


\section{Validación del esquema}
\label{sec:esquema}
[escribe aquí]

\end{document}
