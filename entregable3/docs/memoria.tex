\documentclass[]{article}
\usepackage{graphicx}
\graphicspath{{imagenes/}}
\usepackage[spanish]{babel}
\usepackage[a4paper, top=2.5cm, bottom=2.5cm, left=3cm, right=3cm]{geometry}
\usepackage[hidelinks]{hyperref}
\usepackage[T1]{fontenc}
\usepackage{listings}
\usepackage{xcolor}
\usepackage{float}


\definecolor{miverde}{rgb}{0,0.6,0}
\lstdefinelanguage{cql}{
  morekeywords={SELECT, FROM, WHERE, INSERT, INTO, UPDATE, DELETE, CREATE, TABLE, PRIMARY, KEY, IF, EXISTS, NOT, NULL, AND, OR, SET, USE, VALUES, IN, ALLOW, FILTERING}, % Palabras clave
  sensitive=true, % Si las palabras clave distinguen mayúsculas de minúsculas
  morecomment=[l]--, % Comentarios de línea (prefijados con "--")
  morecomment=[s]{/*}{*/}, % Comentarios de bloque
  morestring=[b]', % Cadenas entre comillas simples
  morestring=[b]" % Cadenas entre comillas dobles
}


\lstdefinestyle{cql}{
    language=cql,
    backgroundcolor=\color{gray!2},     % Color de fondo
    basicstyle=\ttfamily,               % Tipo y tamaño de fuente
    keywordstyle=\color{blue}\bfseries, % Color para palabras clave
    stringstyle=\color{miverde},        % Color para cadenas
    commentstyle=\color{red},           % Color para comentarios
    showspaces=false,                   % No mostrar espacios
    showstringspaces=false,             % No mostrar espacios en las cadenas
    frame=single,                       % Poner un marco alrededor del código
    breaklines=true,                    % Romper las líneas largas
    captionpos=b,                       % Posición del caption
    tabsize=4,                          % Tamaño de las tabulaciones
    escapeinside={\%*}{*)},             % Para incluir código LaTeX en los listings
    morekeywords={self},                 % Palabras clave adicionales
    extendedchars=true,
    inputencoding=utf8
}

%title
\title{Práctica 1} 

\author{Adrián Ferández Galán, César López Mantecón y Manuel Gómez-Plana Rodríguez}

\begin{document}

\begin{titlepage}
    \centering
   \includegraphics[width=0.9\textwidth]{uc3m.jpg} 
    {\Huge Universidad Carlos III\\
    
     \Large Arquitectura de Datos\\
     \vspace{0.5cm}
     Curso 2024-25}
    \vspace{2cm}

    {\Huge \textbf{Práctica 2} \par}
    \vspace{0.5cm}
    {\Large Migración de una base de datos a \texttt{Cassandra} \par}
    \vspace{8cm}

   \textbf{Ingeniería Informática, Cuarto curso}\\
    \vspace{0.2cm} 
    Adrián Fernández Galán       (NIA: 100472182, e-mail: 100472182@alumnos.uc3m.es)\\
    César López Mantecón         (NIA: 100472092, e-mail: 100472092@alumnos.uc3m.es)\\
    Manuel Gómez-Plana Rodríguez (NIA: 100472310, e-mail: 100472310@alumnos.uc3m.es)
    \vspace{0.5cm}

   
    \textbf{Prof .} Lourdes Moreno López\\
    
    \textbf{Grupo: } 81   
    
\end{titlepage}
\newpage

\renewcommand{\contentsname}{\centering Índice}
\tableofcontents

\newpage

\section{Introducción}
\label{sec:introduccion}
En este documento se recoge el desarrollo de la práctica 2 de la asignatura
\textit{Arquitectura de Datos}. En esta práctica se tratará de completar una
migración de una base de datos desde \texttt{MongoDB} a \texttt{Cassandra}.
Además, se computarán nuevas tablas con el fin de permitir el análisis
estadístico, aprovechando las cualidades de \texttt{Cassandra} para el análisis
de datos gracias a su capacidad para la consulta masiva de datos de una misma
columna.

La metodología usada para el desarrollo de este proyecto ha sido la siguiente:
análisis de datos y casos de uso, realización del diseño lógico y físico
orientado a \texttt{Cassandra} e implementación de casos de uso en forma e
consulta. Adicionalmente se empleará la herramienta \texttt{PySpark} para
realizar el primer volcado de datos.
% =======
% Para la realización de esta práctica se busca realizar una migración de datos desde el gestor de bases de datos \textit{Mongodb} a \textit{Cassandra}, además de suplir una serie de casos de uso 
% 
% Esta práctica se basa en la migración de un sistema gestor de expedientes de sanciones que se encontraba en el gestor de bases de datos \textit{Mongodb} y se quiere mover a \textit{Cassandra}, además de implementar nuevos casos de uso. 
% Para ello se realizará un estudio de los datos almacenados en \textit{mongodb} y de los casos de uso propuestos. Tras esto se desarrollará un diseño lógico y físico de la base de datos en \textit{Cassandra} que nos faciliten la creación de las tablas en este mismo gestor de bases de datos y la implementación de las consultas enfocadas a los casos de uso.
% Con las tablas configuradas se insertarán los antiguos datos a través de la herramienta \textit{PySpark}, para finalmente probar el correcto funcionamiento del nuevo sistema.

\section{Modelo de Información del nuevo sistema}
\label{sec:modelo_informacion}

El nuevo sistema será una migración del antiguo sistema por lo que es necesario realizar un estudio de los datos almacenados en el antiguo modelo. Como la anterior gestión de los datos se realizaba en el gestor de bases de datos \textit{mongodb} se tiene un json con todos los datos.

Observando las características del json con la información del anterior sistema podemos sacar las siguientes conclusiones:
\begin{itemize}
    \item El antiguo sistema almacena registro sobre los vehículos que han circulado por diferentes autovías, este registro se ha hecho a través de grabaciones realizadas por los diferentes radares situados por las autovías.
    \item El sistema también gestiona sanciones ya emitidas como multas por velocidad (\textit{speed ticket}), cargos administrativos (\textit{clearance ticket}) y multas extendidas (\textit{stretch ticket})
    \item Para cada uno de los registros se pueden identificar los siguientes elementos:
    \begin{itemize}
        \item Una autovía o carretera
        \item Un radar en un kilometro determinado de la carretera
        \item Un vehículo que ha cruzado el radar
        \item El vehículo consta de información del conductor en ese momento y del propietario del vehículo
        \item Una grabación realizada por el radar sobre el vehículo en cuestión
    \end{itemize} 
\end{itemize}

Con el objetivo de entender el dominio del problema se ha creado un diagrama de clases en formato UML que nos permita entender los elementos que se tienen en el sistema y cómo estos se relacionan entre sí.

\section{Análisis de los Casos de Uso}
\label{sec:analisis_casos_de_uso}

Para poder desarrollar el nuevo sistema es necesario entender a la perfección los casos de uso que se nos plantean, por lo que a continuación se analizarán los distintos casos de uso que se nos presentan.

Los casos de uso se pueden dividir según su finalidad:
\begin{itemize}
    \item \textbf{Por funciones operativas}: Aquellos casos de uso que tienen como objetivo generar nuevas funcionalidades
    \item \textbf{Para análisis estadístico}: Aquellos casos de uso que tienen como objetivo organizar la información de tal manera que pueda realizarse un análisis estadístico sobre ella.
\end{itemize}

\subsection{Funciones Operativas}
\label{subsec:funciones_operativas}

Encontramos 2 casos de uso enfocados a las funciones operativas.
\begin{itemize}
    \item \textbf{Emisión de sanciones}
    
    Este caso de uso busca generar nuevas sanciones a aquellos conductores que no cumplen con una serie de condiciones relacionadas con la velocidad del coche durante el tramo de radar, la situación personal del conductor identificado, el estado de las revisiones técnicas del vehículos y otros factores.
    A continuación se realizará descripción de las condiciones que generarán una nueva sanción:
    \begin{itemize}
        \item \textit{Discrepancia en el carné de conducir del conductor}: En aquellos casos en los que la fecha de obtención del permiso de conducir no sea superior a la fecha de nacimiento del conductor en 18 años.
        \item \textit{Conducción con el vehículo deficiente}: Un coche se considera deficiente si en su última revisión se ha identificado algún defecto. 
        \item \textit{Impago de sanciones emitidas}: Se generará una nueva sanción a aquellos conductores no hayan realizado el pago durante la fecha prevista, dado que no se tiene el conocimiento de esta fecha se tomará una fecha que corresponda con la mediana de todas las fechas existentes en la base de datos.
    \end{itemize}

    Con el objetivo de cumplir este caso de uso se ha generado una tabla llamada \texttt{sanciones}. En esta tabla se recopilarán todas las sanciones de cualquier tipo, tanto las generadas como las emitidas. Para ello se generará la siguiente tabla:
\begin{table}[H]
    \centering
    \begin{tabular}{l} 
        \textbf{Sanciones} \\ \hline
        DNI \hspace{0.6cm}K\\
        Tipo \hspace{0.5cm}C↑\\
        Fecha \hspace{0.4cm}C↓\\
        Estado \\
        Matrícula \\
        Cantidad
    \end{tabular}
    \end{table}

    Tal y como funciona \textit{Cassandra} es necesario crear una tabla vaya de la mano de la consulta. Para satisfacer el caso de uso se realizará a través de la siguiente consulta:
    \begin{lstlisting}
        Select * from Sanciones WHERE DNI = '12345678X' AND Tipo IN ('discrepancia carne', 'desperfectos', 'impago') 
    \end{lstlisting}

    Como se puede observar para poder realizar esta consulta en \textit{Cassandra} es necesario que el \textit{DNI} y \textit{Tipo} formen parte de la clave primaria de la tabla. Además de estos dos atributos se considera que \textit{Fecha} permite ordenar todas las sanciones generadas y concede control en la unicidad de los registros. 
    \item \textbf{Apertura de un proceso ejecutivo para expedientes activos sin plazo de pago cerrado}
    
    Este caso de uso busca identificar aquellos expedientes que aún continuan activos pero su plazo de pago no ha sido cerrado, para posteriormente abrir un proceso ejecutivo que es realizado por otro sistema de gestión de la DGT que permita avisar a los conductores para que realicen su pago.
\end{itemize}

\subsection{Análisis Estadístico}
\label{subsec:analisis_estadistico}

Para el análisis estadístico la DGT está interesada en los siguientes estudios:
\begin{itemize}
    \item Estudio por marcas y modelos

    Este caso de uso busca realizar un estudio sobre las marcas y modelos de coches que más infracciones cometen. Es por ello que se exigen tres análisis:
    
    \begin{enumerate}
        \item Número de multas por marca y modelo del vehículo
        \item Los tres colores más multados de coches
        \item Las marcas y modelos de los vehículos con más infracciones por velocidad
    \end{enumerate}

    Es por ello que se han realizado tres tablas distintas sobre las que hacer las querys en Cassandra que permitan obtener cada una de estadísticas. Las tablas tienen esta forma: [introducir fotos]

    Además, las querys que actúan sobre las tablas tienen este aspecto:

    \lstset{style=cql}
    \begin{lstlisting}[language=cql, caption=Querys para el caso de uso 1]
// Query para el numero de multas por marca y modelo
SELECT Multas FROM multas_marca_modelo WHERE marca="Valor_Marca" AND modelo="Valor_Modelo";

// Query para el numero de multas por color
SELECT Color FROM multas_color_coche ORDER BY (Multas) DESC LIMIT 3;

// Query para sacar las marcas y modelos de los vehiculos con mas infracciones por velocidad
SELECT Marca, Modelo FROM velocidad_marca_modelo ORDER BY (Multas);
    \end{lstlisting}

    \item Estudio por carreteras

    Este caso de uso busca realizar un estudio sobre las carreteras más conflictivas de España, haciendo un análisis del exceso de velocidad medio en carreteras, así como los tramos donde más infracciones se han cometido. Es por ello que se pide lo siguiente:

    \begin{enumerate}
        \item Exceso de velocidad medio para una carretera determinada
        \item Tramo y sentido más conflictivo de una carretera 
    \end{enumerate}

    Para poder hacer las consultas, es necesario dos tablas que se han definido de la siguiente manera [introducir fotos]

    Las querys para obtener las estadísticas son las siguientes:

    \begin{lstlisting}[language=cql, caption=Querys para el caso de uso 2]
// Query para el exceso de velocidad medio para una carretera determinada
SELECT Exceso_velocidad FROM exceso_velocidad_carretera WHERE Carretera = "Valor_Carretera";

// Query para el tramo y sentido mas conflictivo de una carretera
SELECT Kilometro, Sentido FROM conflictos_tramo_sentido WHERE Carretera = "Valor_Carretera" ORDER BY (Infracciones) DESC LIMIT 1;
    \end{lstlisting}
 
    \item Estudio por conductores

    El último caso de uso busca realizar un estudio sobre los conductores, concretamente, buscan determinar los conductores más infractores, así como la probabilidad de que se cometa una infracción cuando un vehículo no es conducido por el dueño del mismo. Para esto, se han diseñados las siguientes tablas: [introducir fotos]

    Para consultar estos valores, se han diseñado las siguientes querys:

    \begin{lstlisting}[language=cql, caption=Querys para el caso de uso 3]
// Query para los conductores mas infractores
SELECT DNI FROM concutores_mas_infractores ORDER BY (Suma_infracciones) DESC;

// Query para la probabilidad
a

    \end{lstlisting}
\end{itemize}

\newpage
\section{Conclusión}
\label{sec:conclusion}

\end{document}
